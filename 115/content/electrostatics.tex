\subsubsection*{Electrostatics}
$\vec{F} = \frac{|q_1 q_2|}{4\pi \varepsilon_0 r^2}$\\
$\varepsilon_0 = 8.85 \times 10^-12 \frac{C^2}{Nm^2}$\\
$e = 1.6 \times 10^{-19} C$\\
$\vec{E} \equiv \frac{\vec{F}}{Q}$ | $[\vec{E}] = \frac{N}{C} = \frac{V}{m}$\\
$\vec{E} = \frac{q}{4\pi\varepsilon_0 r^2}\hat{r}$\\
$\vec{E} = \sum\vec{E}_i = \frac{q}{4\pi\varepsilon_0 r_i^2}\hat{r}_i$\\
Energy in a field $ U_E = \frac{1}{2}\varepsilon_0 E^2$\\
$\vec{E} = \int d\vec{E} =  \frac{1}{4\pi\varepsilon_0}\int\frac{dq}{r^2}\hat{r}$\\
$\Phi = \vec{E}\cdot \vec{A}$
\subsubsection*{Steps for Continuous Charge Distributions}
1. Divide into small charges $dQ$\\
2. Find var for pos of $dQ$ w/ bounds\\
3. Determine field due to $dQ$\\
4. Break field into components if needed\\
5. Express all in terms of const and pos. var\\
6. Integrate wrt pos var
\subsubsection*{Gauss's Law}

$\oint \vec{E}\cdot d\vec{A} = \frac{Q_{enc}}{\varepsilon_0}$\\
Always true but  only useful to find $E$ when symmetrical (typically
spherical, cylindrical, planar)\\
Conductors arrange charge to have zero internal electric field.\\
Charge will always move to the surface of a conductor.\\
$|\vec{E}| = \frac{\sigma}{\varepsilon}$

